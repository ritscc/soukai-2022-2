\subsection*{アドカレ総括}

%\writtenBy{\syogaiChief}{尾﨑}{真央}
\writtenBy{\syogaiStaff}{尾崎}{真央}

毎日記事を投稿できなかった.
メインとなる1・2回生が少なく,記事を確保することが困難だったため,2023年度は3回生の執筆を義務づけることも検討すべきである.
今回はテーマが情報系という広範囲にわたるものであったため,RCCの幅広い活動をアピールすることができた.
RCCの活動を他団体に伝えるという目的は,twitterで多数の人に共有されていたことから,達成されたと考えられる.
渉外局内の目的である「一回生に運営を経験させること」については現渉外局長の引き継ぎが比較的されており,達成されたと考えられる.

