\subsection*{\thirdGrade{}総括}

%\writtenBy{\firstGrade}{駒谷}{亮叡}
%\writtenBy{\secondGrade}{駒谷}{亮叡}
\writtenBy{\thirdGrade}{駒谷}{亮叡}
%\writtenBy{\fourthGrade}{駒谷}{亮叡}

2022年度秋学期の\thirdGrade{}方針は,以下の4点であった.

\begin{enumerate}
    \item 引き継ぎ文書を残す
    \item サークルルームの利用率をできるだけ高く保つ
    \item 行事の運営のサポートをする
    \item 新入生の活動への継続的な参加を促す
\end{enumerate}


\subsubsection*{引き継ぎ文書を残す}
引き継ぎ文書をScrapboxで作成し,下回生が今後の活動を円滑に行うことができるようにした.
ただし,一部項目では文書ではなく口頭などで引き継ぎを行った.

\subsubsection*{サークルルームの利用率をできるだけ高く保つ}

回生を問わず多くのメンバーが日常的にサークルルームを利用していた.
学園祭や局コンパ,プロジェクト活動などでもサークルルームが活用されていたため,会員間の親睦を深めることにも寄与したと言える.

\subsubsection*{行事の運営のサポートをする}

\thirdGrade{}が主導することもあったが,多くは\secondGrade{}主体で運営することができた.
ハッカソンやLT会,新年会など,ほとんどのイベントを対面で開催することができた.

\subsubsection*{新入生の活動への積極的な参加を促す}

\secondGrade{}の人数が少なく支援が不十分な点も存在したが,
\firstGrade{}がリーダーのプロジェクトもあり積極的に活動できていたと言える.
