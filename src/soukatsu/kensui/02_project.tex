\subsection*{プロジェクト総括}

\writtenBy{\kensuiChief}{宮寺}{大樹}
%\writtenBy{\kensuiStaff}{宮寺}{大樹}

本項では本局におけるプロジェクト活動業務に関する2022年度秋学期の総括を以下の点において述べる.

\begin{itemize}
  \item 企画書の募集
  \item 週報の回収・催促
  \item 会員のプロジェクト管理
  \item 発表の機会の提供
  \item 報告書の管理
\end{itemize}

\subsubsection*{企画書の募集}

2022年度春学期に設立した通年プロジェクトが二つで,2022年度秋学期に新しく二つの企画書が提出された.
企画書を局会議と上回生会議で確認を行い,全ての企画書に問題が無かったため,全てのプロジェクトを設立した.

\subsubsection*{週報の回収・催促}

各プロジェクトリーダーは,プロジェクト活動の進捗確認や問題の有無の確認を行うために,
週報の提出が義務付けられている.
週報の回収にはGoogleフォームが用いられ,Slackのリマインダー機能を用いてリマインドが行われた.
概ね週報の提出が遅延することはなかった.

\subsubsection*{会員のプロジェクト管理}

本局は,各会員がどのプロジェクトに所属しているかを把握し,
プロジェクトが途中で終了した場合などに所属していた会員のプロジェクト異動などを管理している.

Iot班は活動に必要不可欠なRaspberry Piが故障してしまったため途中から活動できなかったため,
上回生会議で対応を話し合うべきであった.次回からは注意するべき点である.

また,会員のプロジェクト管理を円滑にするために
入会費の支払い記録とは別に本会の活動に参加する会員が把握できるようなリストを次回から取り入れることが
2021年度秋学期研究推進局プロジェクト総括で記載されていたためGoogleフォームを用いて出席を取れるようにしたが,
周知不足や手間がかかるなどの理由であまり使われなかった.

\subsubsection*{発表の機会の提供}

プロジェクト活動の成果発表をプロジェクト発表会を通じて行った.
発表時間を予め制限していたため,発表は円滑に進行した.

発表ができないプロジェクト班もあったが,報告書は提出されていたため,発表は見送ることにした.

\subsubsection*{報告書の管理}

プロジェクト活動の成果を記録として残すため,各プロジェクトリーダーには報告書の提出を義務としている.
全てのプロジェクト班が報告書を提出し,成果発表に間に合わなかったプロジェクトを除いて,プロジェクト活動の成果発表の際に参加者全員でレビューし,
問題ないことを確認した.
