\subsection*{2022年度秋学期活動総括}

\writtenBy{\president}{Park}{Jooinh}

本会の目的である「情報科学の研究,及びその成果の発表を活動の基本に会員相互の親睦を図り,
学術文化の創造と発展に寄与する」ことを達成するため,方針として以下の五つを立てた.
これらについてそれぞれ評価を行うことで2021年度秋学期の総括とする.

\begin{itemize}
    \item 親睦を深める
    \item 規律ある行動
    \item 自己発信力の向上
    \item 会員間の技術向上
    \item 外部への情報発信
\end{itemize}

\subsubsection*{親睦を深める}
    2022年度秋学期活動では,主にプロジェクト活動の実施や,新年会,局活動の開催を通して会員間の親睦を図った.

    尚,2022年度秋学期活動においては2022年春学期に続き対面活動が許可されていたため,主な活動は対面で行い,
    必要に応じてオンラインでの活動を併用する形をとった.

    一部のプロジェクト班では活動が滞っていたが,その他のプロジェクト班では積極的に議論が行われ,
    例年のプロジェクト活動より遥かに親睦を深めることのできる場となった.

    また,例年行われていたクリスマス会の代わりに企業協賛イベントの新年会を開催した.
    新年会ではコミュニケーションを十分取ることができるように工夫されていたため,
    参加した会員間で親睦を深めることができた.
    
    局活動は限定的に行われていたが,対面で交流する機会もあったことから,
    ある程度親睦を深めることができた.
    
    しかし,対面活動がメインになったことで対面参加が難しい会員の参加率が低くなった側面が見られた.

    一部のプロジェクト班や局では活動が滞っていたが,対面での活動が定着し,
    サークルルームを頻繁に訪れる会員が増えたことから,目的は達成したと言える..

\subsubsection*{規律ある行動}
    2022年度秋学期の方針として,遅刻・欠席連絡と備品整理,
    サークルルームの使用方法の三つの項目からなる行動規範を定めた.
    
    遅刻・欠席連絡は,例年より理由が明確に記載されるようになり,
    全体的に遅れることなく行われていた.
    
    サークルルームの使用方法については,ゴミの処理を怠っていた会員が一部発生していた.
    
    備品整理については,有志の会員により清掃が行われ整頓することができた.
    学生部によるサークルルーム点検においても,特に戒告を受けることはなかった.

\subsubsection*{自己発信力の向上}
    自己発信力を向上させるための機会として,2022年度秋学期活動では,
    LTやプロジェクト発表会,Advent Calendar,会誌の制作を実施した.

    定例会議におけるLTは,参加率が低く,遅延も多かったため例年より消極的に行われた.

    プロジェクト発表会では,一部のプロジェクト班を除きプロジェクトの発表を滞りなく行うことができた.
    
    Advent Calendarは実施することができた一方,記事が不足したことから複数記事を書いた会員も存在した.

    会誌は\secondGrade{}の人数が少ないため\secondGrade{}全体が担当したが,
    担当者が曖昧になったため企画進行が少し遅れた.
    2022年度は学園祭が開催され,会誌の配布を行うこととなった.
    用意していた部数を早々に配布することができ,本会Webサイト上での公開も行った.
    
    これらの活動によって,自己発信力を向上させることにつながったと考えられる.
    
\subsubsection*{会員間の技術向上}
    会全体の技術力を向上させることを目的として,LTやプロジェクト活動を実施した.

    定例会議におけるLTでは先述の通り,例年に比べて参加率が低かったため十分に達成したとは言えない.

    プロジェクト活動は,対面活動が基本になり,質問しやすい環境になっていた.
    特に\firstGrade{}がプロジェクトリーダーになっている班もあり,
    他会員との交流が例年より活発に行うことができた.

    また,技術力向上の場として学園祭ハッカソンを開催した.
    参加者は技術力向上につながったが,\thirdGrade{}の参加が少なかったため
    \firstGrade{}だけでプロダクトを完成させることができなかった.
    コロナの影響で上回生の人数が少なくなっているため対策が必要である.
    
    LTの参加率は低かったがプロジェクトやハッカソンへの参加率は高く,ある一定以上の技術向上がみられた.

\subsubsection*{外部への情報の発信}
    会外へ活動を発信する機会として,主に本会Webサイトと会公式Twitterが挙げられる.

    本会Webサイトでは新年会やLT会などの企業協賛イベントについての記事を投稿した.
    しかし,プロジェクト報告書は期日までに仕上げることができなかったため公開できなかった.

    会公式Twitterでは,先述した企業協賛イベントや学園祭,ハッカソンなどのイベントの様子
    について投稿することができた.
