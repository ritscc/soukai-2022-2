\subsection*{プロジェクト活動総括}

%\writtenBy{\president}{宮寺}{大樹}
%\writtenBy{\subPresident}{宮寺}{大樹}
%\writtenBy{\firstGrade}{宮寺}{大樹}
\writtenBy{\secondGrade}{宮寺}{大樹}
%\writtenBy{\thirdGrade}{宮寺}{大樹}
%\writtenBy{\fourthGrade}{宮寺}{大樹}

\subsubsection*{全体総括}
2022年度秋学期のプロジェクト活動は,
2022年度春学期に設立された通年プロジェクトと2022年度秋学期に設立された半期プロジェクトの両方で行われた.
各プロジェクトには活動ごとに週報を提出することを義務付け,進捗確認を行った.
2022年度秋学期も2022年度春学期から引き続きBCPレベルが2であったため,基本的に対面で活動した.


以下に2022年度秋学期に活動していたプロジェクトの一覧を示す.

\begin{itemize}
  \item DTM班
  \item Unity班
  \item セキュリティ班
  \item Iot班

\end{itemize}

プロジェクト活動の総括は以下の六つに分けて行う.

\begin{itemize}
  \item 目標の総括
  \item プロジェクトの内容
  \item 週報
  \item 報告書
  \item 追い込み合宿
  \item プロジェクト発表会
\end{itemize}

\subsubsection*{目標の総括}
2022年度秋学期の目標は以下の三つであった.

\begin{itemize}
  \item 活動を通して技術力の向上を図る
  \item 個人のみならずグループ活動としての経験を得る
  \item 活動によって得られた成果を本会Webサイトを通して公開する
\end{itemize}

これらを踏まえた総括を以下に記す.

活動を通して技術力の向上を図るに関しては,
程度に差はあれど,プロジェクト活動に参加した会員の技術力が向上したことから,
この目標は概ね達成できたと言える.

集団行動の重要性を学ぶに関しては,
プロジェクト活動の参加率は高く,達成できていたと言える.
しかし,途中からグループでの活動が少なく,個人活動がほとんどとなっていた班も存在した.

得られた成果を本会Webサイトを通して公開するに関しては,
プロジェクト活動報告書は提出されており,程なくWebサイトに公開する予定である.

\subsubsection*{プロジェクトの内容}
プロジェクトの内容については,春学期から引き続き全ての班において適切であった.

\subsubsection*{週報}
2022年度秋学期は週報に翌週に使用したい部屋の希望を聞く項目を追加したため,ほぼすべての週報が提出された.

しかし,Iot班はRaspberryPiが故障していたため活動が滞り実質解散状態であった.
これについては上回生会議で解散について議論するべきであった.

\subsubsection*{報告書}

報告書の提出をもってプロジェクト終了としており,
全ての班が報告書を提出したことにより,
これを達成できた.
報告書の内容に関しては,ほぼ問題はなかった.
報告書の形式については,PDF形式で提出するものとした.

\subsubsection*{追い込み合宿}
昨年追い込み合宿で進捗を出していたことから追い込み合宿は十分な有用性があると言えが,スケジュールなどを考慮して今学期は行わなかった.

\subsubsection*{プロジェクト発表会}

プロジェクト活動を通して得ることができた知見や技術を会内で共有する場として,オンラインでプロジェクト発表会を行った.

プロジェクト発表会は報告書のレビューを行う時間をとった後に発表,質疑応答を行うといった形式で行った.

レビューは5分間時間をとり,時間が足りなければ追加時間をとるといった形式で行った.

発表ができなかった班が一つ存在したが,報告書は提出されていたため,
発表は見送ることにした.
