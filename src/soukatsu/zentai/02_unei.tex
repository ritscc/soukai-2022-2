\subsection*{運営総括}

%\writtenBy{\president}{山本}{京介}
\writtenBy{\subPresident}{山本}{京介}
%\writtenBy{\firstGrade}{山本}{京介}
%\writtenBy{\secondGrade}{山本}{京介}
%\writtenBy{\thirdGrade}{山本}{京介}
%\writtenBy{\fourthGrade}{山本}{京介}

\subsubsection*{運営サポート}
秋学期は\secondGrade{}中心で運営を行うが,\thirdGrade{}もサポートするという方針を立てたが,
局活動や上回生会議などで達成できていた.

\subsubsection*{定例会議}
定例会議は毎週開催するという方針を立てたが,ほぼ毎週の木曜日に開催できていた.
会内向けのフォームは積極的にform viewerを活用する方針を立てたが,一部フォームでは失念により活用されていなかった.

\subsubsection*{上回生会議}
上回生会議は毎週開催するという方針を立てた.方針通り毎週開催するということは達成したが,執行部の不手際で時間通りに開催できない日もあった.
必要に応じて臨時で開催されることもあった.
上回生会議には各局の局長及びその代理人が出席することになっているが,一部の局が欠席していることがあった.
なお,\firstGrade{}の出席はなかった.

\subsubsection*{企画}
秋学期には学園祭,新年会,企業LT会といった企画が開催された.
学園祭は主に学園祭担当が企画を進めていき,上回生会議で進捗確認を行った.
また,開催後にはKPTを行った.
その他の企画については,企画の詳細を上回生会議にて議論を行った.
