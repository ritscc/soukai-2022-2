\subsection*{\newGradeIfKouki{}\thirdGrade{}方針}

\writtenBy{\secondGrade}{古川}{聡悟}

2022年度は2021年度とは異なり,対面での活動を行うことができた.
2023年度は2022年度と同様,\newGradeIfKouki{}\thirdGrade{}が\newGradeIfKouki{}\secondGrade{}のサポートを積極的に行う方針とする.
運営の主体は\newGradeIfKouki{}\secondGrade{}であるが,行事についてサポートが必要な場合は,対応時期の確認を一緒に行うなどのサポートを\newGradeIfKouki{}\thirdGrade{}が心がける.
また,対面のコミュニケーションのしやすさとオンラインの気軽さの両方を活用しつつ,\newGradeIfKouki{}\thirdGrade{}として会内の活動に積極的に参加する.
さらに,サークルの知名度や内容の周知のためポスター,イベント,展示や動画を使い新歓を行う.
オンラインの有効活用として,教え合いや小規模な勉強会の開催といったDiscordでの交流を増やす.
定例会議をZoomで質問できる仕組み,LT内容を予習できるサイトの告知やZoomでコメントを流すことやアンケートによる日時の調整により改善する.
個人技術の向上を促すために,本や機材の購入申請や貸し借りをしやすくする.
具体的には申請者は購入理由のスライドを作り,定例会議中に公開するだけで申請者自体は登壇しないようにする.
決議の際には,投票の匿名化のためオンライン上で決議を取る.
購入申請をする前に先輩に聞くことができるチャンネルをSlack上に設ける.
