\subsection*{財務方針}

%\writtenBy{\kaikeiChief}{佐久間}{智也}
\writtenBy{\kaikeiStaff}{冨高}{麟太郎}

2023年度春学期財務方針として下記の六つを示す.
\begin{itemize}
    \item 会計情報の公開
    \item 会計ガイダンス
    \item 購入申請
    \item 入会手続き・会費
    \item 春学期基盤活動助成金
    \item LTアンケート景品
\end{itemize}

\subsection*{会計情報の公開}
定例会議,上回生会議において必要とされる場合において予算,執行額,予算執行率を Googleドライブ上で公開する.
また,より詳細な情報は開示要求があった場合に\kaikeiStaff{}を通じて公開する.

\subsection*{会計ガイダンス}
会員に対して購入申請の方法,領収書の切り方, 3Dプリンターと印刷用プリンターについての説明を目的としたガイダンスを行う.
また,このガイダンスで会費の使用用途,学友会費についても説明を行うこととする.

\subsection*{購入申請}
活動状況を考慮したうえで購入申請を2022年度秋学期と同様に受け付ける.

\subsection*{入会手続き・会費}
新規入会者は,Googleフォームへの入力と会費の支払いを行った後に入会届の発行を受け,署名をする.
その後,執行部が書類を一時受理,保存する.\president{}に受理された後,入会手続きは完了する.
本会への所属経験のあるものはGoogleフォームへの入力と会費を支払う.
会員から6000円を会費として入会届と同時に徴収する.
対面での徴収が難しい場合,口座振り込みなどの他の手段で代替する.

\subsection*{春学期基盤活動助成金}
学友会費予算を考慮したうえで申請を行うかどうかの決定をする.

\subsection*{LTアンケートの景品}
2022年度のLT会の優勝者には景品として定例会議での審査を免除した書籍の購入申請権を与えていたが,
参加者の意欲向上や商品の汎用性を広げるために新しい景品で代替することとする.
