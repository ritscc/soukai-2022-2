\subsection*{学園祭方針}

%\writtenBy{\president}{梶原}{悠人}
%\writtenBy{\subPresident}{梶原}{悠人}
\writtenBy{\firstGrade}{梶原}{悠人}
%\writtenBy{\secondGrade}{梶原}{悠人}
%\writtenBy{\thirdGrade}{梶原}{悠人}
%\writtenBy{\fourthGrade}{梶原}{悠人}

\subsubsection*{目的}
学園祭に参加する目的は以下の通りである.
\begin{itemize}
    \item 学術部公認団体としての還元活動
    \item 本会の能力向上
    \item サークル外部への情報発信
\end{itemize}
これらを目的に学園祭を運営する.

\subsubsection*{目標}
目的の達成のために以下の2点を目標とする.
\begin{itemize}
    \item 来場者数150名
    \item 会誌頒布数120部
\end{itemize}
これらに加えてプロジェクトの発表と制作物の作成を行うことで会員の能力向上を図る.

\subsection*{企画}
学園祭の企画は以下の内容を行う.
\begin{itemize}
    \item プロジェクト活動発表
    \item プロジェクトポスター
    \item 制作物の展示
    \item アンケート
\end{itemize}

\paragraph{プロジェクト活動発表}
2022年度同様\firstGrade{},\secondGrade{}が担当する
一般の方に伝わるようわかりやすくする.
当タイムキーパーは担当者を振り分け,発表時間を5分~15分に収める.
発表後に質疑応答をし,質問に対して答えれるよう回答を事前に用意する.  

\paragraph{プロジェクトポスター}
夏休み中にポスターを制作する.
プロジェクトリーダーが主体として作成する.
その場にいる会員が対応できるよう他のプロジェクトの内容も把握しておく.

\paragraph{制作物の展示}
プロジェクト活動とは別に\secondGrade{},\thirdGrade{}は各自の制作物の提出を義務づける.
他回生の提出は任意とする.

\paragraph{アンケート}
2019年度のものを踏襲して作成する.

\subsubsection*{広告物}
広告物に載せる内容を決定させるために制作物を早めに確定させる.
展示している内容がわかるよう.
学園祭での広告は以下の手段で行う.
\begin{itemize}
    \item ビラ
    \item 会公式Twitter
    \item ポスター
    \item 動画
    \item 本会Webサイト
    \item 人間広告
\end{itemize}

\paragraph{ビラ}
大学から許可が出た場合行う.
担当者はビラを配る人を把握する.
なるべく夏休み中に作成する.
ビラは100枚固定で印刷する.

\paragraph{会公式Twitter}
2022年度同様に渉外局が担当する.
一週間まえから事前に告知し,学園祭当日開催しているイベントについて紹介する.

\paragraph{ポスター}
掲示板に張り出し,学園祭終了後は回収する.
ポスター制作は担当を決めず,会内でコンペティションを行う
ポスターに記載する内容は事前に会議で決定する.

\paragraph{動画}
会内で動画担当を募集する.
webサイトで公開し,当日会場でも流す.

\paragraph{本会Webサイト}
Twitter同様渉外局が担当する.
トップページを学園祭仕様に置き換える.

\paragraph{人間広告}         
ビラ同様に大学の許可がある場合行う.
人間広告班を設立し,班員が看板を作成する.

\subsubsection*{会場}
机が移動式の教室を希望する.
部屋の希望は以下の通りである.
\begin{itemize}
    \item 第1希望:プリズムハウス
    \item 第2希望:ラルカディア
    \item 第3希望:コラーニングハウスⅡ
\end{itemize}
部屋のレイアウトは部屋が決定する前に求められるため,決定後必要に応じてレイアウトを変更する.

\subsubsection*{シフト}
夏休み明けに決定する.
発表と人間広告が連続にならないようにする.
制作物作成者はできるだけ担当する.

\subsubsection*{KPT}
学園祭後の定例会議で行う.
学園祭当日までに担当者がドキュメントを公開し,次の定例会議までに全員が記入してで全体確認をする.

